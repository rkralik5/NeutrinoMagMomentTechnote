\RequirePackage{lineno}
\documentclass[12pt]{article}
\usepackage[utf8]{inputenc}
\usepackage{fullpage}
\usepackage[T1]{fontenc}
\usepackage{amsmath, amsthm, amssymb,amsfonts}
\usepackage{mathptmx}
\usepackage{graphicx}
\usepackage{authblk} % To add authors' affiliations

\usepackage{hyperref}
\hypersetup{colorlinks,
            citecolor=blue,
            filecolor=black,
            linkcolor=black,
            urlcolor=black
}

\title{Measurement of the neutrino magnetic moment at the NOvA experiment\\ \vspace*{1cm} \texttt{Technical note}}
\author[1]{Robert Kralik}
\affil[1]{University of Sussex, Brighton, UK}
\date{\today}

%\linenumbers % Include line numbers

\begin{document}
\maketitle

\begin{abstract}
    This is the abstract
\end{abstract}

\section{Introduction}

%The same types of experimental measurements are also sensitive to more exotic neutrino electromagnetic properties: magnetic moments and millicharges, which would be certainly due to new physics beyond the Standard Model. The discovery of millicharges or anomalously large neutrino magnetic moments would have also important implications for astrophysics and cosmology. [NeutrinoPropertiesSnowmass2022.pdf]

%neutrino electromagnetic interactions [...] provide powerful tools to probe the physics beyond the standard model. ... Hence, the theoretical and experimental study of neutrino electromagnetic interactions is a powerful tool in the search for the fundamental theory beyond the standard model. Moreover, the electromagnetic interactions of neutrinos can generate important effects, especially in astrophysical environments, where neutrinos propagate over long distances in magnetic fields in vacuum and in matter. Unfortunately, in spite of many efforts in the search of neutrino electromagnetic interactions, up to now there is no positive experimental indication in favor of their existence. However, it is expected that the standard model neutrino charge radii should be measured in the near future. This will be a test of the standard model and of the physics beyond the standard model which contributes to the neutrino charge radii. Moreover, the existence of neutrino masses and mixing implies that neutrinos have magnetic moments. Since their values depend on the specific theory which extends the standard model in order to accommodate neutrino masses and mixing, experimentalists and theorists are eagerly looking for them. [nuElmagInt2015.pdf]

%The importance of neutrino electromagnetic properties was first mentioned by Pauli in 1930, when he postulated the existence of this particle and discussed the possibility that the neutrino might have a magnetic moment (Pauli, W., 1991, Cambridge Monogr. Part. Phys., Nucl. Phys., Cosmol. 1, 1.). [nuElmagInt2015.pdf]

%Systematic theoretical studies of neutrino electromagnetic properties started after it was shown that in the extended standard model with right-handed neutrinos the magnetic moment of a massive neutrino is, in general, nonvanishing and that its value is determined by the neutrino mass (Lee and Shrock, 1977; Marciano and Sanda, 1977; Petcov, 1977; Fujikawa and Shrock, 1980; Pal and Wolfenstein, 1982; Shrock, 1982; Bilenky and Petcov, 1987). Neutrino electromagnetic properties are important because they are directly connected to fundamentals of particle physics. For example, neutrino electromagnetic properties can be used to distinguish Dirac and Majorana neutrinos, because Dirac neutrinos can have both diagonal and offdiagonal magnetic and electric dipole moments, whereas only the off-diagonal ones are allowed for Majorana neutrinos (Schechter and Valle, 1981; Kayser, 1982, 1984; Nieves, 1982; Pal and Wolfenstein, 1982; Shrock, 1982). Another important case in which Dirac and Majorana neutrinos have quite different observable effects is the spin-flavor precession in an external magnetic field discussed in Sec. VI.B. Neutrino electromagnetic properties are also probes of new physics beyond the standard model, because in the standard model neutrinos can have only a charge radius. The discovery of other neutrino electromagnetic properties would be a signal of new physics beyond the standard model (Bell et al., 2005, 2006; Bell, 2007; Novales-Sanchez et al., 2008). [nuElmagInt2015.pdf]

%Considering an experiment which does not observe any effect of neutrino magnetic mometn and obtains a limit on the neutrino mag. moment, it is possible to obtain, following Studenikin (2014), a bound on neutrino millicharge by demanding that the effect of neutrino millicharge is smaller that tha of neutrino magnetic moment. [nuElmagInt2015.pdf - page 580]

\subsection{Theoretical overview}
\subsection{Experimental overview}
\subsection{Analysis overview}
\subsection{Datasets and Event Reconstruction details}
\subsubsection{Enhanced $\nu$-on-e sample}
\subsection{Analysis weights}
\subsubsection{Neutrino magnetic moment as a weight}

\section{Systematic uncertainties}
\subsection{Neutrino flux systematics}
\subsection{Detector systematics}
\subsection{Cross section systematics}

\section{Event selection}

\section{Resolution and binning}

\bibliographystyle{unsrturl}
\bibliography{literature}
\end{document}
