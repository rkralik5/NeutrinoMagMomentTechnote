\section{Theoretical overview}

%Neutrino electromagnetic properties have been proposed since the very beginning by Pauli to solve the discrepancies in the electron beta emission spectra. This was solved by discovering the neutron. Then again, neutrino magnetic moment was proposed as one of the solution to the solar neutrino problem 

% Although in the standard model neutrinos are electrically neutral and do not possess electric or magnetic dipole moments, they have a charge radius which is generated by radiative corrections. [...] In many extensions of the standard model neutrinos also acquire electromagnetic properties through quantum loop effects which allow direct interactions of neutrinos with electromagnetic fields and electromagnetic interactions of neutrinos with charged particles. Hence, the theoretical and experimental study of neutrino electromagnetic interactions is a powerful tool in the search for the fundamental theory beyond the standard model. Moreover, the electromagnetic interactions of neutrinos can generate important effects, especially in astrophysical environments, where neutrinos propagate over long distances in magnetic fields in vacuum and in matter. [nuElmagInt2015.pdf]

% ...the existence of neutrino masses and mixing implies that neutrinos have magnetic moments. Since their values depend on the specific theory which extends the standard model in order to accommodate neutrino masses and mixing, experimentalists and theorists are eagerly looking for them. [nuElmagInt2015.pdf]

%Systematic theoretical studies of neutrino electromagnetic properties started after it was shown that in the extended standard model with right-handed neutrinos the magnetic moment of a massive neutrino is, in general, nonvanishing and that its value is determined by the neutrino mass (Lee and Shrock, 1977; Marciano and Sanda, 1977; Petcov, 1977; Fujikawa and Shrock, 1980; Pal and Wolfenstein, 1982; Shrock, 1982; Bilenky and Petcov, 1987). [nuElmagInt2015.pdf]

%Neutrino electromagnetic properties are important because they are directly connected to fundamentals of particle physics. For example, neutrino electromagnetic properties can be used to distinguish Dirac and Majorana neutrinos, because Dirac neutrinos can have both diagonal and off-diagonal magnetic and electric dipole moments, whereas only the off-diagonal ones are allowed for Majorana neutrinos (Schechter and Valle, 1981; Kayser, 1982, 1984; Nieves, 1982; Pal and Wolfenstein, 1982; Shrock, 1982). This is shown in detail in Secs. III.A and III.B. Another important case in which Dirac and Majorana neutrinos have quite different observable effects is the spin-flavor precession in an external magnetic field discussed in Sec. VI.B. Neutrino electromagnetic properties are also probes of new physics beyond the standard model, because in the standard model neutrinos can have only a charge radius (see Secs. III.C and VII.B). The discovery of other neutrino electromagnetic properties would be a signal of new physics beyond the standard model (Bell et al., 2005, 2006; Bell, 2007; Novales-Sanchez et al., 2008). [nuElmagInt2015.pdf]

In the Standard Model (SM), neutrinos are massless and electrically neutral particles. However, even in the SM neutrinos can have electromagnetic interaction through loop diagrams involving the charged leptons and the W boson. These interactions are described by the neutrino charge radius, described in section \ref{sec:otherNuElmagProperties} \cite{NeutrinoPropertiesSnowmass2022.pdf}.

%But this is only the neutrino charge radius, not the neutrino electric or magnetic moment (maybe also the anapole moment) "Hence, in the standard model the form factor can be interpreted as a neutrino charge radius or as an anapole moment (or as a combination of both). The standard model theory of the neutrino charge radius has a long history, with some controversies which are shortly summarized in the following." [nuElmagInt2015.pdf - sec.VIIB]

%Various theories beyond the Standard Model
To include neutrino masses required by neutrino oscillations, we must go Beyond the Standard Model (BSM), where neutrinos can acquire other electromagnetic properties \cite{nuElmagInt2015.pdf}. In the most general case, considering interactions with a single photon as shown on Fig.\ref{fig:Feynman}, neutrino electromagnetic interactions can be described by an \textit{effective} interaction Hamiltonian \cite{nuElmagInt2015.pdf}
\begin{equation}
\mathcal{H}^{\left(\nu\right)}_{em}\left(x\right)=\sum^N_{k,j=1}\overline{\nu}_k\left(x\right)\Lambda^{kj}_{\mu}\nu_j\left(x\right)A^{\mu}\left(x\right).
\end{equation}
Here $\nu_k\left(x\right), k\in\left\lbrace 1,...,N\right\rbrace$ are neutrino fields in the mass basis with $N$ neutrino mass states. $\Lambda^{kj}_{\mu}$ is a general vertex function and $A^{\mu}\left(x\right)$ is the electromagnetic field.

\begin{figure}
\centering
\begin{tikzpicture}
  \begin{feynman}
    \vertex[draw,circle] (m) at ( 0, 0) {$\Lambda^{fi}$};
    \vertex (a) at (-2,1) {\(\nu_i(p_i)\)};
    \vertex (b) at (2,1) {\(\nu_f(p_f)\)};
    \vertex (c) at (0,-1.5) {\(\gamma(q)\)};

    \diagram* {
      (a) -- [fermion] (m) -- [fermion] (b),
      (m) -- [boson] (c)
    };
  \end{feynman}
\end{tikzpicture}
\caption{Effective coupling of neutrinos with one photon electromagnetic field.}
\label{fig:Feynman}
\end{figure}

\iffalse
The amplitude of neutrino-to-neutrino interaction for \textbf{Dirac} neutrinos is
\begin{equation}
\braket{\nu_f\left(p_f\right)|j^{\left(\nu\right)}_{\mu}\left(x\right)|\nu_i\left(p_i\right)}=
e^{i\left(p_f-p_i\right)x}\overline{u}_f\left(p_f\right)\Lambda^{fi}_{\mu}\left(p_f,p_i\right)u_i\left(p_i\right),
\end{equation}
where $p_f$ and $p_i$ are the final and initial four momentums respectively and $u/\overline{u}$ are the solutions to the Dirac equation for a free particle. We take into account possible transitions between different mass states $\nu_i$ and $\nu_f$ \cite{nuElmagInt2015.pdf}. \todo{also describe what is j}
\fi

The vertex function $\Lambda^{fi}_{\mu}\left(q\right)$ is generally a matrix and in the most general case can be written in terms of linearly independent products of Dirac matrices $\left(\gamma\right)$ and only depends on the square of the four momentum of the photon $\left(q=p_f-p_i\right)$:
\begin{align}
\Lambda^{fi}_{\mu}\left(q\right)=&
\mathbb{F}^{fi}_1\left(q^2\right)q_{\mu}+
\mathbb{F}^{fi}_2\left(q^2\right)q_{\mu}\gamma_5+
\mathbb{F}^{fi}_3\left(q^2\right)\gamma_{\mu}+
\mathbb{F}^{fi}_4\left(q^2\right)\gamma_{\mu}\gamma_5+\notag\\ &
\mathbb{F}^{fi}_5\left(q^2\right)\sigma_{\mu\nu}q^{\nu}+
\mathbb{F}^{fi}_6\left(q^2\right)\epsilon_{\mu\nu\rho\gamma}q^{\nu}\sigma^{\rho\gamma},
\end{align}
where $\mathbb{F}^{fi}_i\left(q^2\right)$ are six Lorentz invariant form factors \cite{nuElmagInt2015.pdf}.

Applying conditions of hermiticity $\left(\mathcal{H}^{\left(\nu\right)\dagger}_{em}=\mathcal{H}^{\left(\nu\right)}_{em}\right)$ and of the gauge invariance of the electromagnetic field, we can rewrite the vertex function as
\begin{equation}
\Lambda^{fi}_{\mu}\left(q\right)=
\left(\gamma_{\mu}-q_{\mu}\slashed{q}/q^2\right)\left[
\mathbb{F}^{fi}_{Q}\left(q^2\right)+\mathbb{F}^{fi}_{A}\left(q^2\right)q^2\gamma_5\right]-
i\sigma_{\mu\nu}q^{\nu}\left[\mathbb{F}^{fi}_{M}\left(q^2\right)+i\mathbb{F}^{fi}_{E}\left(q^2\right)\gamma_5\right],
\end{equation}
where $\mathbb{F}^{fi}_Q,\mathbb{F}^{fi}_M,\mathbb{F}^{fi}_E$ and $\mathbb{F}^{fi}_A$ are hermitian matrices representing the charge, dipole magnetic, dipole electric and anapole neutrino form factors. In coupling with a real photon $\left(q^2=0\right)$ these become the neutrino charge and magnetic, electric and anapole moments. The neutrino charge radius corresponds to the second term in the expansion of the charge form factor \cite{nuElmagInt2015.pdf}.

We can simplify the above expression as \cite{NeutrinoPropertiesSnowmass2022.pdf}
\begin{equation}
\Lambda^{fi}_{\mu}\left(q\right)=\gamma_{\mu}\left(Q_{\nu_{fi}}+\frac{q^2}{6}\langle r^2\rangle_{\nu_{fi}}\right)-i\sigma_{\mu\nu}q^{\nu}\mu_{\nu_{fi}},
\end{equation}
where $Q_{\nu_{fi}}$, $\langle r^2\rangle_{\nu_{fi}}$, and $\mu_{\nu_{fi}}$ are the neutrino charge, effective charge radius (also containing anapole moment), and an effective magnetic moment (also containing electric moment) respectively. This is possible thanks to the proportional effect of the neutrino charge radius and the anapole moment, or the neutrino magnetic and electric moment respectively \cite{nuElmagInt2015.pdf}. These quantities (charge, charge radius and magnetic moment) are the three neutrino electromagnetic properties measured in experiments.

\iffalse
For antineutrinos the form factors are transformed as:
\begin{equation}\label{eqAnu1}
\overline{\mathbb{F}}^{fi}_{\Omega}=-\mathbb{F}^{if}_{\Omega}=-\left(\mathbb{F}^{fi}_{\Omega}\right)^{\star} \ \ \ \Omega=Q,M,E,
\end{equation}
\begin{equation}\label{eqAnu2}
\overline{\mathbb{F}}^{fi}_{A}=\mathbb{F}^{if}_{A}=\left(\mathbb{F}^{fi}_{A}\right)^{\star}.
\end{equation}
\todo{maybe describe what does this mean?}

In case of \textbf{Majorana neutrinos}, the general expression for the vertex function in terms of charge, magnetic, electric and anapole form factors looks the same as for Dirac neutrinos.
\todo{so does that mean that the interaction amplitude can be written in the same way for both Dirac and Majorana neutrinos?} However, since Majorana antineutrinos are the same particle as Majorana neutrinos, from eq.\ref{eqAnu1},\ref{eqAnu2} we can see that:
\begin{equation}\label{eqAntisymmetryCondition}
\mathbb{F}^M_{\Omega}=-\left(\mathbb{F}^M_{\Omega}\right)^T \ \ \ \Omega=Q,M,E,
\end{equation}
\begin{equation}
\mathbb{F}^M_{A}=\left(\mathbb{F}^M_A\right)^T.
\end{equation}
Therefore the Majorana charge, magnetic and electric form factor matrices are antisymmetric and the anapole form factor matrix is symmetric. This means that Majorana neutrino doesn't have any diagonal charge and dipole magnetic and electric moments, but it can have transition  charge and magnetic and electric moment \cite{nuElmagInt2015.pdf}.
\todo{Explain why is this worth mentioning or remove it if it's not}
\fi

%%%%%%%%%%%%%%%%%%%%%%%%%%%%%%%%%%%%%%%%
\subsection{Neutrino electric and magnetic dipole moments}

The size and effect of the neutrino electromagnetic properties depends on the specific theory beyond standard model.

Evaluating the one loop diagrams in the minimal extension of the standard model \todo{should I mention here what actually do I mean by the minimal extension of SM? probably this:  "introduction of three right-handed neutrinos"} with right handed (Dirac) neutrinos gives us the first approximation of the electric and magnetic moments $\left(q^2=0\right)$:
\begin{equation}
\begin{rcases}
\mu^D_{kj}\\
i\epsilon^D_{kj}
\end{rcases}
\simeq\frac{3eG_F}{16\sqrt{2}\pi^2}\left(m_k\pm m_j\right)\left(\delta_{kj}-\frac{1}{2}\sum_{l=e,\mu ,\tau}U^{\star}_{lk}U_{lj}\frac{m_l^2}{m_W^2}\right),
\end{equation}
where $m_k,m_j$ are the neutrino masses and $m_l$ are the masses of charged leptons which appear in the loop diagrams. Higher order electromagnetic corrections were neglected, but those can also have a significant contribution \cite{nuElmagInt2015.pdf}.
\todo{ok, so what does that mean? Does that mean that this equation is not actually correct or that it's the higher/lower limit? or something else?}

(It can be seen that) There are no diagonal electric moments $\left(\epsilon_{kk}^D=0\right)$ and the diagonal magnetic moments are approximately
\begin{equation}\label{DiagMagMomVal}
\mu_{kk}^D\simeq\frac{3eG_Fm_k}{8\sqrt{2}\pi^2}\simeq 3.2\times 10^{-19}\left(\frac{m_k}{\textsf{eV}}\right)\mu_B,
\end{equation}
where $\mu_B$ is the Bohr magneton \cite{nuElmagInt2015.pdf}.

The transition magnetic moments are suppressed with respect to the largest of the diagonal magnetic moments by at least a factor of $10^{-4}$ due to the $m_W^2$ in denominator and the transition electric moments are even smaller than that due to the mass difference \cite{nuElmagInt2015.pdf}. Therefore an experimental observation of a magnetic moment larger than in eq.\ref{DiagMagMomVal} would indicate physics beyond the minimally extended standard model \cite{nuMMMajoranaBounds2006.pdf}.

Majorana neutrinos can be obtained by either adding a $\textsf{SU}\left(2\right)_L$ Higgs triplet, or right handed neutrinos together with a $\textsf{SU}\left(2\right)_L$ Higgs singlet \todo{is this also considered a minimally extended SM? or is this the only way to get Majorana neutrinos? Most likely the former - from nuElmagInt2015.pdf: the absence of Higgs triplets, without which it is not possible to have Majorana mass terms.}. If we neglect the Feynman diagrams which depend on the model of the scalar sector \todo{what does that mean for the result?}, the magnetic and electric dipole moments are
\begin{equation}
\mu_{kj}^M\simeq -\frac{3ieG_F}{16\sqrt{2}\pi^2}\left(m_k+m_j\right)\sum_{l=e,\mu ,\tau}\operatorname{Im}\left[U^{\star}_{lk}U_{lj}\right]\frac{m_l^2}{m_W^2},
\end{equation}
\begin{equation}
\epsilon_{kj}^M\simeq \frac{3ieG_F}{16\sqrt{2}\pi^2}\left(m_k-m_j\right)\sum_{l=e,\mu ,\tau}\operatorname{Re}\left[U^{\star}_{lk}U_{lj}\right]\frac{m_l^2}{m_W^2}.
\end{equation}
These are difficult to compare to the Dirac case, due to possible presence of Majorana phases in the PMNS matrices, but it is clear that they have the same order of magnitude as Dirac transition dipole moments. However, the neglected model dependent contributions can enhance the transition dipole moments \cite{nuElmagInt2015.pdf}. \todo{but how much?}

It is possible \cite{nuMMMajoranaBounds2006.pdf} to obtain "natural" upper limits on the size of neutrino magnetic moment by calculating its contribution to the neutrino mass by standard model radiative corrections. For Dirac neutrinos the radiative correction induced by neutrino magnetic moment, generated at an energy scale $\Lambda$, to the neutrino mass is generically
\begin{equation}
m_{\nu}^D\sim\frac{\mu_{\nu}^D}{3\times 10^{-15}\mu_B}\left[\Lambda\left(\textsf{TeV}\right)\right]^2\textsf{eV}.
\end{equation}
So for $\Lambda\simeq 1\textsf{TeV}$ and $m_{\nu}\lesssim 0.3\textsf{eV}$ the limit becomes $\mu_{\nu}^D\lesssim 10^{-15}\mu_B$. This applies only if the new physics is well above the electroweak scale ($\Lambda_{EW} \sim 100\textsf{GeV}$). It is possible to get Dirac neutrino magnetic moment higher than this limit, for example in frameworks of minimal super-symmetric standard model, by adding more Higgs doublets, or by considering large extra dimensions \cite{nuElmagInt2015.pdf}.

The limit for Majorana neutrino magnetic moment is less stringent, due to the antisymmetry condition from eq.\ref{eqAntisymmetryCondition} and considering $m_{\nu}\lesssim 0.3\textsf{eV}$ can be expressed as
\begin{align}
\mu_{\tau\mu},\mu_{\tau e} &\lesssim 10^{-9}\left[\Lambda\left(\textsf{TeV}\right)\right]^{-2}\\
\mu_{\mu e} &\lesssim 3\times 10^{-7}\left[\Lambda\left(\textsf{TeV}\right)\right]^{-2}
\end{align}
which is shown in the flavour basis , which relates to the framework used previously as
\begin{equation}
\mu_{ij}=\sum_{\alpha\beta}\mu_{\alpha\beta}U^{\star}_{\alpha i}U_{\beta j},\ \ \ \alpha,\beta\in\left\lbrace e,\mu,\tau\right\rbrace.
\end{equation}
This limits imply, that if a magnetic moment $\mu\gtrsim 10^{-15}\mu_B$ would be measured, it is plausible neutrinos are Majorana fermions and the scale of lepton violation would be well below the conventional see-saw scale \cite{nuMMMajoranaBounds2006.pdf}.

\subsection{Other neutrino electromagnetic properties}\label{sec:otherNuElmagProperties}

Neutrino electric charge is heavily constraint by the measurements on the neutrality of matter (since generally neutrinos having an electric charge would also mean that neutrons have charge which would affect all heavier nuclei). It is also constrained by the SN1987A, since neutrino having an effective charge would lengthen its path through the extragalactic magnetic fields and would arrive on earth later. It can also be obtained from nu-on-e scatter from the relationship between neutrino millicharge and magnetic moment. [nuElmagInt2015.pdf - sec. VIIA]

The neutrino charge radius is determined by the second term in the expansion of the neutrino charge form factor and can be interpreted using the Fourier transform of a spherically symmetric charge distribution. It can also be negative since the charge density is not a positively defined quantity. In the SM the charge radius has the form of (possible other definitions exist)
\begin{equation}
\langle r_{\nu_l}^2\rangle_{\textsf{SM}}=\frac{G_{\textsf{F}}}{4\sqrt{2}\pi^2}\left[3-2\log\left(\frac{m_l^2}{m_W^2}\right)\right].
\end{equation}
This corresponds to $\langle r_{\nu_{\mu}}^2\langle_{\textsf{SM}}=2.4\times 10^{-33}\ \unit{cm^2}$ and similar scale for other neutrino flavours. [nuElmagInt2015.pdf - sec. VIIB]

[nuElmagInt2015.pdf - sec. VIIB]
The effect of the neutrino charge radius on the neutrino-on-electron scattering cross section is through the following shift of the vector coupling constant (Grau and Grifols, 1986; Degrassi, Sirlin, and VMarciano, 1989; Vogel and Engel, 1989; Hagiwara et al., 1994):
\begin{equation}
g_V^{\nu_l}\rightarrow g_V^{\nu_l}+\frac{2}{3}m_W^2\langle r_{\nu_l}^2\rangle\sin^2\theta_W
\end{equation}

[nuElmagInt2015.pdf - sec. VIIB]
The current experimental limits for muon neutrinos are from \todo{check the current exp. limits}  Hirsch, Nardi, and Restrepo (2003) who obtained the
following 90\% C.L. bounds on $\langle r_{\nu_\mu}^2\rangle$ from a reanalysis of
CHARM-II (Vilain et al., 1995) and CCFR (McFarland et al.,1998) data:
\begin{equation}
-0.52\times 10^{-32}<\langle r_{\nu_\mu}^2\rangle<0.68\times 10^{-32}\ \unit{cm^2}
\end{equation}

In the Standard Model, the neutrino anapole moment is somehow coupled with the neutrino charge radii and is functionally identical. the phenomenology of neutrino anapole moments is similar to that of neutrino charge radii. Hence, the limits on the neutrino charge radii discussed in Sec. VII.B also apply to the neutrino anapole moments multiplied by 6.  in the standard model the neutrino charge radius and the anapole moment are not defined separately and one can interpret arbitrarily the charge form factor as a charge radius or as an anapole moment. Therefore, the standard model values for the neutrino charge radii in Eqs. (7.35)–(7.38) can be interpreted also as values of the corresponding neutrino anapole moments. [nuElmagInt2015.pdf - sec. VIIC]

It is possible to consider  the toroidal dipole moment as a characteristic of the neutrino which is more convenient and transparent than the anapole moment for the description of T-invariant interactions with nonconservation of the P and C symmetries. the toroidal and anapole moments coincide in the static limit when the masses of the initial and final neutrino states are equal to each other. The toroidal (anapole) interactions of a Majorana as well as a Dirac neutrino are expected to contribute to the total cross section of neutrino elastic scattering off electrons, quarks, and nuclei. Because of the fact that the toroidal (anapole) interactions contribute to the helicity preserving part of the scattering of neutrinos on electrons, quarks, and nuclei, its contributions to cross sections are similar to those of the neutrino charge radius. In principle, these contributions can be probed and information about toroidal moments can be extracted in low-energy scattering experiments in the future. Different effects of the neutrino toroidal moment are discussed by Ginzburg and Tsytovich (1985), Bukina, Dubovik, and Kuznetsov (1998a, 1998b), and Dubovik and Kuznetsov (1998). In particular, it has been shown that the neutrino toroidal electromagnetic interactions can produce Cherenkov radiation of neutrinos propagating in a medium. [nuElmagInt2015.pdf - sec. VIIC]

%%%%%%%%%%%%%%%%%%%%%%%%%%%%%%%%%%%%%%%%%%%%%%%%

\subsection{Measuring neutrino magnetic moment}
%Maybe use "Effect of neutrino magnetic moment on measurement" instead
\todo{Need to add some general description of the measurements}
\subsubsection{Effective neutrino magnetic moment}
\todo{Describe why experiments measure an effective magnetic moment}
What we measure in experiments is an effective "flavour" magnetic moment, which is influenced by mixing of "mass" magnetic moments (and electric moments) and oscillations. In the ultrarelativistic limit this is
\begin{equation}
\mu_{\nu_l}^2\left(L,E_{\nu}\right)=\sum_j\left|\sum_k U^{\star}_{lk}e^{-i\Delta m^2_{kj}L/2E_{\nu}}\left(\mu_{jk}-i\epsilon_{jk}\right)\right|^2.
\end{equation}
What is called the effective magnetic moment (often just magnetic moment) therefore contains contributions from both the neutrino magnetic and electric moment \cite{nuElmagInt2015.pdf}.

For antineutrinos, the effective magnetic moment is
\todo{maybe I should just combine these two equations to avoid overcrowding}
\begin{equation}
\mu_{\overline{\nu}_l}^2\left(L,E_{\nu}\right)=\sum_j\left|\sum_k U^{\star}_{lk}e^{+i\Delta m^2_{kj}L/2E_{\nu}}\left(\mu_{jk}-i\epsilon_{jk}\right)\right|^2.
\end{equation}
So the only difference is in the phase induced by neutrino oscillations.

For experiments with baselines short enough for neutrino oscillations to not develop ($\frac{\Delta m^2L}{2E_{\nu}}\ll~1$), such as the NOvA ND, the effective magnetic moment can be expressed as
\begin{equation}
\mu_{\nu_l}^2\simeq\mu_{\overline{\nu}_l}^2\simeq\sum_j\left|\sum_k U_{lk}^{\star}\left(\mu_{jk}-i\epsilon_{jk}\right)\right|^2=\left[U\left(\mu^2+\epsilon^2\right)U^{\dagger}+2\operatorname{Im}\left(U\mu\epsilon U^{\dagger}\right)\right]_{ll^{\prime}},
\end{equation}
which is independent of the neutrino energy and of the source to detector distance.

It is important to mention, that since the effective magnetic moment depends on the flavour of the studied neutrino, it is different for different types of neutrino experiment. Also the solar neutrino experiments need to include the effect of the solar matter on the neutrino oscillations. Therefore the reports on the value (or upper limit) of the effective neutrino magnetic moment are not directly comparable between different types of neutrino experiments.

\subsubsection{Neutrino-on-electron elastic scattering}
The most sensitive method to measure neutrino magnetic moment is the low energy elastic scattering of (anti)neutrinos on electrons \cite{nuElmagInt2015.pdf}. This interaction has two observables, the recoil electron's kinetic energy $\left(T_e\right)$ and the recoil angle with respect to the incoming neutrino beam $\left(\theta\right)$. From simple $2\rightarrow 2$ kinematics we can get
\begin{equation}
\left(P_{\nu}-P_{e^{\prime}}\right)^2=\left(P_{\nu^{\prime}}-P_e\right)^2,
\end{equation}
\begin{equation}
m_{\nu}^2+m_e^2-2E_{\nu}E_{e^{\prime}}+2E_{\nu}p_{e^{\prime}}\cos\theta=m_{\nu}^2+m_e^2-2E_{\nu^{\prime}}m_e.
\end{equation}
Using the energy conservation
\begin{equation}
E_{\nu}+m_e=E_{\nu^{\prime}}+E_{e^{\prime}}=E_{\nu^{\prime}}+T_e+m_e\Rightarrow E_{\nu^{\prime}}=E_{\nu}-T_e
\end{equation}
we get
\begin{equation}
E_{\nu}p_{e^{\prime}}\cos\theta=E_{\nu}E_{e^{\prime}}-E_{\nu^{\prime}}m_e=E_{\nu}\left(T_e+m_e\right)-\left(E_{\nu}-T_e\right)m_e=T_e\left(E_{\nu}+m_e\right),
\end{equation}
\begin{equation}
\cos\theta=\frac{E_{\nu}+m_e}{E_{\nu}}\sqrt{\frac{T_e^2}{E_{e^{\prime}}^2-m_e^2}}=\frac{E_{\nu}+m_e}{E_{\nu}}\sqrt{\frac{T_e^2}{T_e^2+2T_em_e}}.
\end{equation}
And finally we get
\begin{equation}\label{eqThetaTRelation}
\cos\theta=\frac{E_{\nu}+m_e}{E_{\nu}}\sqrt{\frac{T_e}{T_e+2m_e}}.
\end{equation}
Electron's kinetic energy is kinematically constrained by
\begin{equation}
T_e\leq\frac{2E_{\nu}^2}{2E_{\nu}+m_e}.
\end{equation}

Considering $E_{\nu}\sim\textsf{GeV}$, we can approximate $\frac{m_e^2}{E_{\nu}^2}\rightarrow 0$ and in the small angle approximation we get from eq.\ref{eqThetaTRelation} 
\begin{equation}\label{eqTThetaSqExp}
T\theta^2\cong 2m_e\left(1-\frac{T_e}{E_{\nu}}\right)<2m_e.
\end{equation}

In the ultrarelativistic limit, the neutrino magnetic moment changes the neutrino helicity, turning active neutrinos into sterile \todo{this is a very strong statement and it probably need a bit more backing up}. Since the SM weak interaction conserves helicity we can add the two contribution to the neutrino on electron cross section incoherently \cite{nuElmagInt2015.pdf}:
\begin{equation}
\frac{d\sigma_{\nu_le^-}}{dT_e}=\left(\frac{d\sigma_{\nu_le^-}}{dT_e}\right)_{\textsf{SM}}+\left(\frac{d\sigma_{\nu_le^-}}{dT_e}\right)_{\textsf{MAG}}.
\end{equation}

The standard model contribution can be expressed as \cite{nuElmagInt2015.pdf}:
\begin{equation}
\left(\frac{d\sigma_{\nu_le^-}}{dT_e}\right)_{\textsf{SM}}=\frac{G_F^2m_e}{2\pi}\left\lbrace\left(g_V^{\nu_l}+g_A^{\nu_l}\right)^2+\left(g_V^{\nu_l}-g_A^{\nu_l}\right)^2\left(1-\frac{T_e}{E_{\nu}}\right)^2+\left(\left(g_A^{\nu_l}\right)^2-\left(g_V^{\nu_l}\right)^2\right)\frac{m_eT_e}{E_{\nu}^2}\right\rbrace,
\end{equation}
where the coupling constants $g_V$ and $g_A$ are different for different neutrino flavours and for antineutrinos. Their values are:
\begin{align}
g_V^{\nu_e}&=2\sin^2\theta_W+1/2,\hspace{2.5cm} g_A^{\nu_e}=1/2,\\
g_V^{\nu_{\mu,\tau}}&=2\sin^2\theta_W-1/2,\hspace{2.25cm} g_A^{\nu_{\mu,\tau}}=-1/2.
\end{align}
For antineutrinos $g_A\rightarrow -g_A$.

Using expressions \ref{eqThetaTRelation} and \ref{eqTThetaSqExp} we can also derive \cite{NuOnECrossSections1989.pdf} cross sections with respect to $\cos\theta$, $\theta^2$ and $T\theta^2$:
\todo{I don't think these equations are actually valid. We've been using some approximations and therefore I don't think these equations work for any theta dependence}

\begin{multline}
\left(\frac{d\sigma_{\nu_le^-}}{d\cos\theta}\right)_{\textsf{SM}}=
\frac{2G_F^2E_{\nu}^2m_e^2\cos\theta\left(E_{\nu}+m_e\right)^2}{\pi\left(\left(E_{\nu}+m_e\right)^2-E_{\nu}^2\cos^2\theta\right)^2}\\
\left\lbrace\left(g_V^{\nu_l}+g_A^{\nu_l}\right)^2 +
\left(g_V^{\nu_l}-g_A^{\nu_l}\right)^2\left(1-\frac{2m_eE_{\nu}\cos^2\theta}{\left(E_{\nu}+m_e\right)^2-E_{\nu}^2\cos^2\theta}\right)^2\right. +\\
\left.\left(\left(g_A^{\nu_l}\right)^2-\left(g_V^{\nu_l}\right)^2\right)
\frac{2m_e^2\cos^2\theta}{\left(\left(E_{\nu}+m_e\right)^2-E_{\nu}^2\cos^2\theta\right)}\right\rbrace,
\end{multline}
 
\begin{multline}
\left(\frac{d\sigma_{\nu_le^-}}{d\theta^2}\right)_{\textsf{SM}}=
\frac{G_F^2m_e^2}{\pi\left(\theta^2+\frac{2m_e}{E_{\nu}}\right)^2}
\left\lbrace
\left(g_V^{\nu_l}+g_A^{\nu_l}\right)^2+\left(g_V^{\nu_l}-g_A^{\nu_l}\right)^2
\left(\frac{\theta^2}{\theta^2+\frac{2m_e}{e_{\nu}}}\right)^2\right. +\\
\left.\left(\left(g_A^{\nu_l}\right)^2-\left(g_V^{\nu_l}\right)^2\right)
\frac{2m_e^2}{E_{\nu}^2\left(\theta^2+\frac{2m_e}{E_{\nu}}\right)}\right\rbrace,
\end{multline}

\begin{multline}
\left(\frac{d\sigma_{\nu_le^-}}{dT\theta^2}\right)_{\textsf{SM}}=
\frac{G_F^2E_{\nu}}{4\pi}
\left\lbrace
\left(g_V^{\nu_l}+g_A^{\nu_l}\right)^2+\left(g_V^{\nu_l}-g_A^{\nu_l}\right)^2
\left(\frac{T\theta^2}{2m_e}\right)^2\right. +\\
\left.\left(\left(g_A^{\nu_l}\right)^2-\left(g_V^{\nu_l}\right)^2\right)
\frac{m_e}{E_{\nu}}\left(1-\frac{T\theta^2}{2m_e}\right)\right\rbrace.
\end{multline}

The neutrino magnetic moment contribution is \todo{include derivation from \cite{NeutrinoElmagFormFactors1989.pdf}} \cite{nuElmagInt2015.pdf}:
\begin{equation}
\left(\frac{d\sigma_{\nu_le^-}}{dT_e}\right)_{\textsf{MAG}}=\frac{\pi\alpha^2}{m_e^2}\left(\frac{1}{T_e}-\frac{1}{E_{\nu}}\right)\left(\frac{\mu_{\nu_l}}{\mu_B}\right)^2,
\end{equation}
where $\alpha$ is the fine structure constant.

Analogically to previous, we can also express this cross section in $\cos\theta$, $\theta^2$ and $T\theta^2$:
\begin{equation}
\left(\frac{d\sigma_{\nu_le^-}}{d\cos\theta}\right)_{\textsf{MAG}}=
\frac{2\pi\alpha^2\left(E_{\nu}+m_e\right)^2}{m_e^2\cos\theta}
\frac{\left(E_{\nu}+m_e\right)^2-E_{\nu}^2\cos^2\theta-2m_eE_{\nu}\cos^2\theta}{\left(\left(E_{\nu}+m_e\right)^2-E_{\nu}^2\cos^2\theta\right)^2}
\left(\frac{\mu_{\nu_l}}{\mu_B}\right)^2,
\end{equation}

\begin{equation}
\left(\frac{d\sigma_{\nu_le^-}}{d\theta^2}\right)_{\textsf{MAG}}=\frac{\pi\alpha^2}{m_e^2}\frac{\theta^2}{\left(\theta^2+\frac{2m_e}{E_{\nu}}\right)}\left(\frac{\mu_{\nu_l}}{\mu_B}\right)^2,
\end{equation}

\begin{equation}
\left(\frac{d\sigma_{\nu_le^-}}{dT\theta^2}\right)_{\textsf{MAG}}=\frac{\pi\alpha^2}{4m_e^4}\frac{T\theta^2}{\left(1-\frac{T\theta^2}{2m_e}\right)}\left(\frac{\mu_{\nu_l}}{\mu_B}\right)^2.
\end{equation}

The magnetic moment contribution exceeds the standard model contribution for low enough $T_e$ \cite{nuElmagInt2015.pdf}:
\begin{equation}
T_e\lesssim\frac{\pi^2\alpha^2}{G_F^2m_e^3}\left(\frac{\mu_{\nu}}{\mu_B}\right)^2\simeq 2.9\times 10^{19}\left(\frac{\mu_{\nu}}{\mu_B}\right)^2\left[\textsf{MeV}\right],
\end{equation}
which does not depend on the neutrino energy and makes neutrino experiment sensitive to lower energetic neutrinos more sensitive to the neutrino magnetic moment.
\todo{this is probably not about sensitivity to lower energetic neutrinos but to lower energetic electron, isn't it?}

\subsubsection{Neutrino on nucleus scattering}

\subsubsection{Cosmological effects}

[NuMMBasicsAndAstro\_2022.pdf]
One of the most important astrophysical consequences of neutrino non-zero effective magnetic moments is the neutrino helicity change $\nu_l\rightarrow\nu_R$ with the appearance of nearly sterile right-handed neutrinos $\nu_R$. In general, this phenomena can proceed in three different mechanisms:
\begin{enumerate}
\item the helicity change in the neutrino magnetic moment scattering on electrons (or protons and neutrons),
\item the neutrino spin and spin-flavour precession in an external magnetic field, and
\item the neutrino spin and spin-flavour precession in the transversally moving matter currents or in the transversally polarized matter at rest
\end{enumerate}
For completeness note that the important astrophysical consequence of nonzero neutrino millicharges is the neutrino deviation from the rectilinear trajectory.
\todo{find out if this is the same thing that IceCube describes in their paper}